% XeLaTeX can use any Mac OS X font. See the setromanfont command below.
% Input to XeLaTeX is full Unicode, so Unicode characters can be typed directly into the source.

% The next lines tell TeXShop to typeset with xelatex, and to open and save the source with Unicode encoding.

%!TEX TS-program = xelatex
%!TEX encoding = UTF-8 Unicode

\documentclass[12pt]{paper}
\usepackage{geometry}                % See geometry.pdf to learn the layout options. There are lots.
\usepackage{CJK}

\geometry{letterpaper}                   % ... or a4paper or a5paper or ... 
%\geometry{landscape}                % Activate for for rotated page geometry
%\usepackage[parfill]{parskip}    % Activate to begin paragraphs with an empty line rather than an indent
\usepackage{graphicx}
\usepackage{amssymb}

% Will Robertson's fontspec.sty can be used to simplify font choices.
% To experiment, open /Applications/Font Book to examine the fonts provided on Mac OS X,
% and change "Hoefler Text" to any of these choices.

\usepackage{fontspec,xltxtra,xunicode}
\defaultfontfeatures{Mapping=tex-text}
\setromanfont[Mapping=tex-text]{Hoefler Text}
\setsansfont[Scale=MatchLowercase,Mapping=tex-text]{Gill Sans}
\setmonofont[Scale=MatchLowercase]{Andale Mono}
\newfontfamily{\J}[Scale=0.85]{Osaka}
\newfontfamily{\C}[Scale=1.0]{STHeiti}

\title{\C 论文标题}
\author{\C 张华}
%\date{}                                           % Activate to display a given date or no date



\begin{document}
\maketitle

% For many users, the previous commands will be enough.
% If you want to directly input Unicode, add an Input Menu or Keyboard to the menu bar 
% using the International Panel in System Preferences.
% Unicode must be typeset using a font containing the appropriate characters.
% Remove the comment signs below for examples.

% \newfontfamily{\A}{Geeza Pro}
% \newfontfamily{\H}[Scale=0.9]{Lucida Grande}


% Here are some multilingual Unicode fonts: this is Arabic text: {\A السلام عليكم}, this is Hebrew: {\H שלום}, 
{\C
	关于面部五官替换技术可行性

从技术上来说,为了达到能让玩家可以定制五官的角色,需要独立的一个个制作好眼睛,嘴巴,眉毛的样式,然后于运行时,在玩家的选择下即使替换相应的五官部件到人物面部上,需要程序员做具有替换机制的控制程序.
从流程上来说,制作一个眼睛或嘴巴,或眉毛却不能简单的独立制作,只有把它们组合起来,跟着面部一起在3D内容制作工具(Maya, 3DsMax)中预览到整体效果,才能不断的修正和制作.
所以,从以上这对矛盾来说,能参数化控制(数值调整)的五官调整方案才是优选的,其次才是mesh或texture/uv的替换,这需要美术与程序的互相配合.

通过对<<九州天空城>>的逆向
1.眼球可以用参数方式调整高光方向,表面反射率,而其周围的睫毛肯定要美术人员来做独立的双面小模型,眼纹使用另一层tatoo纹理来叠加
2.嘴型和唇色可以选择通过预生成纹理的方法来组合嘴型与面部纹理,其对应的diffuse + normal + specular纹理中,可以考虑只修改diffuse,这需要美术先做好整体效果,然后指定面部纹理上一个嘴巴对应区域内一个小纹理的独立导出,然后多个这样的嘴巴部位的纹理,于游戏载入的某个阶段,提前生成组合好的面部整体的纹理,与玩家选择面部时,整体替换使用,这么一来兼顾了性能.其Unity中的实现,需要用到Texture2D.GetPixels(...), Texture2D.SetPixels(...)来替换纹理的局部.
}
and here's some Japanese: {\J 今日は}.
{\C 您好!}


\end{document}  